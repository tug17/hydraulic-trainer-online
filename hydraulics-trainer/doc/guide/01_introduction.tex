\section{Introduction}

\subsection{Architecture}

\verb+ezProbs+ is basically a web application where various problems can be listed.
Numerical values can be configured using sliders.

\subsubsection{Problems}

A problem consists of the following sections:

\begin{description}
  \item[Description] general description of the problem
  \item[Plot] optional plot of the current calculated result
  \item[Parameters] configurable parameters for the problem
  \item[Solution] description of the calculation
\end{description}

\subsection{Requirements}

On a Debian system the following packages must be installed:

\begin{itemize}
  \item \verb+python3-scipy+
  \item \verb+python3-numpy+
  \item \verb+python3-pandas+
  \item \verb+python3-matplotlib+
  \item \verb+python3-flask+
  \item \verb+python3-svgwrite+
\end{itemize}

\subsection{Development}

The following commands are used to start the development server:

\begin{verbatim}
export FLASK_ENV=development
export FLASK_APP=ezprobs
flask run
\end{verbatim}

\subsection{Configuration}

The following parameters can be set in the \verb+config.ini+:

\begin{itemize}
  \item \verb+server.secret_key+ a string to set the password with which
    cookies are enctrypted on the client
  \item \verb+application.submit_on_change+ a boolean to controll whether the
    calculation should be kicked off once a parameter slider is changed
\end{itemize}

